%
%
% Tikz: draw an envelope
\newsavebox\envelope
\savebox{\envelope}{

\newlength\mylen
\setlength\mylen{2cm}
\begin{tikzpicture}[scale=.65]

\coordinate (A) at (0,0);
\coordinate (B) at (1.41\mylen,-\mylen);
\clip
  ([xshift=-0.5\pgflinewidth,yshift=0.5\pgflinewidth]A) --
  ([xshift=0.5\pgflinewidth,yshift=0.5\pgflinewidth]A-|B) --
  ([xshift=0.5\pgflinewidth,yshift=-0.5\pgflinewidth]B) --
  ([xshift=-0.5\pgflinewidth,yshift=-0.5\pgflinewidth]B-|A) --
  ([xshift=-0.5\pgflinewidth,yshift=0.5\pgflinewidth]A);
\draw[mybg,fill=mygr,line cap=rect]
  (A) -- (A-|B) -- (B) -- (B-|A) -- (A);
\draw[mybg]
  (B-|A) -- (0.705\mylen,-.3\mylen) -- (B);
\draw[mybg,fill=mygr!80!mybg,rounded corners=15pt]
  (A) -- (0.705\mylen,-0.6\mylen) -- (A-|B);
\node[anchor=north]
  at ($ (A)!0.5!(B|-A) $ ) {\parbox{\mylen}{}};
\draw[mybg] (A) -- (B|-A);
\end{tikzpicture}
}
%
% output
\newsavebox\sentoutputsmall
\savebox{\sentoutputsmall}{%
\begin{tikzpicture}[scale=.5,text=myfg,font=\small]
 \draw[rounded corners,myfg,very thick] (-.5, -.5) rectangle (.5, 2.5) {};

 \node[label={[label distance=.1cm]0:positive}] (pos) at (0, 2) {};
 \node[label={[label distance=.1cm]0:neutral}] (neu) at (0, 1) {};
 \node[label={[label distance=.1cm]0:negative}] (neg) at (0, 0) {};

 \draw[fill=mybg!10!mygr] (pos) circle[radius=9pt];
 \draw[fill=mybg!90!mygr] (neu) circle[radius=9pt];
 \draw[fill=mybg!90!mygr] (neg) circle[radius=9pt];
 \end{tikzpicture}
}
%
\newsavebox\sentoutput
\savebox{\sentoutput}{%
    \begin{tikzpicture}[scale=.75,text=myfg]
 \draw[rounded corners,myfg,very thick] (-.5, -.5) rectangle (.5, 2.5) {};

 \node[label={[label distance=.25cm]0:positive}] (pos) at (0, 2) {};
 \node[label={[label distance=.25cm]0:neutral}] (neu) at (0, 1) {};
 \node[label={[label distance=.25cm]0:negative}] (neg) at (0, 0) {};

 \draw[fill=mybg!10!mygr] (pos) circle[radius=9pt];
 \draw[fill=mybg!90!mygr] (neu) circle[radius=9pt];
 \draw[fill=mybg!90!mygr] (neg) circle[radius=9pt];
 \end{tikzpicture}
 }
%
% cartoon structure
\newcommand{\cartoon}[2][1]{%
\begin{tikzpicture}%
\node[draw=none, minimum size=#1*1cm, regular polygon, regular polygon sides=5] (p) {};
%
\foreach \i/\j in {#2}%
{
    \draw[hlcolor, ultra thick] (p.corner \i) -- (p.corner \j);
}
%
\foreach \i in {1, ..., 5}%
{
    \draw[myfg,fill=mybg,very thick] (p.corner \i) circle[radius=#1*5pt];
}
\end{tikzpicture}}
\newcommand{\cartoonDense}[1][1]{%
\begin{tikzpicture}%
\node[draw=none, minimum size=#1*1cm, regular polygon, regular polygon sides=5] (p) {};
%
\draw[hlcolor, ultra thick, opacity=.8] (p.corner 1) -- (p.corner 2);
\draw[hlcolor, ultra thick, opacity=.5] (p.corner 1) -- (p.corner 3);
\draw[hlcolor, ultra thick, opacity=.7] (p.corner 1) -- (p.corner 4);
\draw[hlcolor, ultra thick, opacity=.4] (p.corner 1) -- (p.corner 5);
\draw[hlcolor, ultra thick, opacity=.6] (p.corner 2) -- (p.corner 3);
\draw[hlcolor, ultra thick, opacity=.3] (p.corner 2) -- (p.corner 4);
\draw[hlcolor, ultra thick, opacity=.9] (p.corner 2) -- (p.corner 5);
\draw[hlcolor, ultra thick, opacity=.2] (p.corner 3) -- (p.corner 4);
\draw[hlcolor, ultra thick, opacity=.6] (p.corner 3) -- (p.corner 5);
\draw[hlcolor, ultra thick, opacity=.4] (p.corner 4) -- (p.corner 5);
%
\foreach \i in {1, ..., 5}%
{
    \draw[myfg,fill=mybg,very thick] (p.corner \i) circle[radius=#1*5pt];
}
\end{tikzpicture}}
\newcommand{\cartoonSparse}[1][1]{%
\begin{tikzpicture}%
\node[draw=none, minimum size=#1*1cm, regular polygon, regular polygon sides=5] (p) {};
%
\draw[hlcolor, ultra thick, opacity=1] (p.corner 1) -- (p.corner 4);
\draw[hlcolor, ultra thick, opacity=.5] (p.corner 2) -- (p.corner 5);
\draw[hlcolor, ultra thick, opacity=.5] (p.corner 1) -- (p.corner 5);
%
\foreach \i in {1, ..., 5}%
{
    \draw[myfg,fill=mybg,very thick] (p.corner \i) circle[radius=#1*5pt];
}
\end{tikzpicture}}
%
\newcommand{\setupsimplexbary}[1][3.3]{%
\coordinate (L1) at (0:0);
\coordinate (L2) at (0:#1);
\coordinate (L3) at (60:#1);

\node[label=east:{\small$\triangle$}] at (L2) {};

\fill[colorPolytope,opacity=.15]  (L1) -- (L2) -- (L3) -- cycle;
\draw[very thick,colorPolytope]   (L1) -- (L2) -- (L3) -- cycle;

\draw[colorPolytope,fill] (L1) circle[radius=3pt];
\draw[colorPolytope,fill] (L2) circle[radius=3pt];
\draw[colorPolytope,fill] (L3) circle[radius=3pt];
}

\newcommand{\drawcs}{%
\node[anchor=south] at (0, \vecheight*4) {$z=1$};
\node[anchor=south] at (0, \vecheight*3) {$z=2$};
\node[anchor=south] at (0, \vecheight*1.5) {$\cdots$};
\node[anchor=south] at (0, \vecheight*0) {$z=N$};}

\newcommand{\drawscores}[1][\s]{%
\node[anchor=south] at (-1-.5*\vecwidth, \vecheight*5+.1) {$#1$};
\draw[elem,fill=vecfg!60!vecbg] (-1-\vecwidth, \vecheight*4) rectangle (-1, \vecheight*5);
\draw[elem,fill=vecfg!85!vecbg] (-1-\vecwidth, \vecheight*3) rectangle (-1, \vecheight*4);
\draw[elem,fill=vecfg!60!vecbg] (-1-\vecwidth, \vecheight*2) rectangle (-1, \vecheight*3);
\draw[elem,fill=vecfg!75!vecbg] (-1-\vecwidth, \vecheight*1) rectangle (-1, \vecheight*2);
\draw[elem,fill=vecfg!50!vecbg] (-1-\vecwidth, \vecheight*0) rectangle (-1, \vecheight*1);
}

\newcommand{\drawargmax}[1][\p]{%
\node[anchor=south] at (1+.5*\vecwidth, \vecheight*5+.1) {$#1$};
\draw[elem,fill=vecfg! 0!vecbg]  (1, \vecheight*4) rectangle (1+\vecwidth, \vecheight*5);
\draw[elem,fill=vecfg!70!vecbg]  (1, \vecheight*3) rectangle (1+\vecwidth, \vecheight*4);
\draw[elem,fill=vecfg! 0!vecbg]  (1, \vecheight*2) rectangle (1+\vecwidth, \vecheight*3);
\draw[elem,fill=vecfg! 0!vecbg]  (1, \vecheight*1) rectangle (1+\vecwidth, \vecheight*2);
\draw[elem,fill=vecfg! 0!vecbg]  (1, \vecheight*0) rectangle (1+\vecwidth, \vecheight*1);
}
\newcommand{\drawsoftmax}[1][\p]{%
\node[anchor=south] at (1+.5*\vecwidth, \vecheight*5+.1) {$#1$};
\draw[elem,fill=vecfg!30!vecbg]  (1, \vecheight*4) rectangle (1+\vecwidth, \vecheight*5);
\draw[elem,fill=vecfg!50!vecbg]  (1, \vecheight*3) rectangle (1+\vecwidth, \vecheight*4);
\draw[elem,fill=vecfg!35!vecbg]  (1, \vecheight*2) rectangle (1+\vecwidth, \vecheight*3);
\draw[elem,fill=vecfg!25!vecbg]  (1, \vecheight*1) rectangle (1+\vecwidth, \vecheight*2);
\draw[elem,fill=vecfg!15!vecbg]  (1, \vecheight*0) rectangle (1+\vecwidth, \vecheight*1);
}

%% Utilities to draw certain structures more easily

\newcommand{\postag}[3]{%
\begin{tabular}{c c c}%
{\color{hlcolor}\footnotesize #1}&%
{\color{hlcolor}\footnotesize #2}&%
{\color{hlcolor}\footnotesize #3}\\%
dog & on & wheels\\\end{tabular}}

\newcommand{\simpledep}[2]{%
\begin{dependency}[edge style={hlcolor,very thick},hide label,arc edge]%
\begin{deptext}[column sep=.3cm]#1\end{deptext}%
#2%
\end{dependency}}

\newcommand{\matching}[1]{%
\begin{tikzpicture}%
\node[anchor=east] at (0, 1) (s1) {dog};
\node[anchor=east] at (0, 0.5) (s2) {on};
\node[anchor=east] at (0, 0) (s3) {wheels};
%
\node[anchor=west] at (0.6, 1) (t1) {hond};
\node[anchor=west] at (0.6, 0.5) (t2) {op};
\node[anchor=west] at (0.6, 0) (t3) {wielen};
%
\foreach[count=\i] \j in {#1}%
\draw[hlcolor,very thick,
      {Circle[length=1mm,width=1mm]}-{Circle[length=1mm, width=1mm]}]
      (s\i.east) -- (t\j.west);
\end{tikzpicture}}

